\chapter{Contexte général}
\section*{Introduction}
Text, text, text, text, text, text, text, text, text, text, text, text, text, text, text, text, text, text, text, text, text, text, text, text, text, text, text, text, text, text, text, text, text, text, text, text, text...

% Une section

% Exemple d'une section qui porte une référence à une bibliographie
% NB: il faut bien suivre le syntaxe pour ne pas tomber dans le cas où il y a une référence dans la table des matières.
\section[Organisme d'accueil]{Présentation de l’organisme d’accueill \cite{webArticle1}}
\subsection{La Société NeoXam}

% NB: il faut annoncer la figure dabord.
% Pour faire appel à une figure, il suffit d'utiliser le label comme suit :
NeoXam est un leader des logiciels financiers, fournissant des solutions et des services à plus de 150 clients dans 25 pays à travers le monde. NeoXam est attaché au succès de ses clients, fournissant des solutions fiables et évolutives, traitant plus de 14 billions de dollars d'actifs par jour et desservant plus de 10 000 utilisateurs. Grâce à ses talents combinés et à son approche transparente, NeoXam aide les acteurs des achats et des ventes à faire face aux changements continus du secteur des marchés financiers, à se développer et à mieux servir leurs clients. NeoXam compte sur plus de 500 employés avec des bureaux à Paris, Francfort, Luxembourg, Zurich, Genève, Milan, New York, Boston, Hong Kong, Shanghai, Pékin, Tunis et Le Cap.


%space btween parg and logo , you can also use \ hspace for horizontal space %
\vspace*{1cm} 

\begin{figure}[htpb]
\centering
\frame{\includegraphics[width=0.5\columnwidth]{Logo_Entreprise}}
\caption{Logosl NeoXam}
\label{fig:logo_tt}
\end{figure}



\begin{subsection}{Services}
\begin{itemize}[label=\textbullet,font=\normalsize]
\item  \textbf{Modèle ASP NeoXam} propose plusieurs solutions via un modèle ASP sécurisé: NeoXam GP3, Client Portal, NeoXam Density et NeoXam DataHub.
\item \textbf{Aide à la réglementation}  L'expérience et l'expertise des experts de NeoXam couvrent de nombreux domaines tels que les réglementations européennes et chinoises (IFRS 9, Solvabilité II, UCITS 4, SFT), les lacunes locales, les références d'entreprise et les données de marché, et peuvent traiter des véhicules d'investissement complexes de l'avant vers l'arrière Bureau.
\item \textbf{ Services professionnels} Respecter les délais et le budget est la devise des équipes de conseil et de mise en œuvre de NeoXam. Découvrez la gamme de nos services professionnels.
\end{itemize} 

\end{subsection}








\section{\'Etude et critique de l'éxistant}
\section{Problématique}
\section{\'Etude bibliographique (sur un thème précis)}

%On peut ajouter un tableau en utilisant le syntaxe suivant:
%% Ce syntaxe est modifié afin de vous offrir un tableau divisible automatiquement dans les pages.

Le tableau \ref{tab:myfirstlongtable} présente un exemple d'un tableau qui peut être divisé automatiquement dans les pages.

\begin{longtable}[c]{
    |p{.20\textwidth}
    |p{.60\textwidth}|
}
    \caption{Tableau long}
    \label{tab:myfirstlongtable}\\
    \hline
    
    0
    & 0 \\
    \hline 
    
    1
    & 1 \\
    \hline 
    
    2
    & 2 \\
    \hline
    
    3
    & 3 \\
    \hline
    
    4
    & 4 \\
    \hline
    
    5
    & 5 \\ \hline
    
    6
    & 6 \\ \hline
    
    7
    & 7 \\
    \hline
    
    8
    & 8 \\
    \hline
    
    9
    & 9 \\
    \hline
    
    10
    & 10 \\
    \hline
    
    11
    & 11 \\
    \hline
    
    12
    & 12 \\
    \hline
\end{longtable}

\section{Solution proposée et objectifs globaux du projet}

\section{Choix méthodologique}

\section*{Conclusion}
    Conclusion partielle ayant pour objectif de synthétiser le chapitre et d’annoncer le chapitre suivant.